
%%%%%%%%%%%%%%%%%%%%%%%%%%%%%%%%%%%%%%%%%%%%%%%%%%%%%%%%%%%%%%%%%%
% Sample template for MIT Junior Lab Student Written Summaries
% Available from http://web.mit.edu/8.13/www/Samplepaper/sample-paper.tex
% Last Updated April 12, 2007
% Adapted from the American Physical Societies REVTeK-4 Pages
% at http://publish.aps.org

\setlength{\paperheight}{11in}
% http://tex.stackexchange.com/questions/74636/mla-package-and-thumbpdf
\makeatletter
\@namedef{ver@thumbpdf.sty}{}
\makeatother
\documentclass[aps,secnumarabic,balancelastpage,amsmath,amssymb,nofootinbib]{revtex4}

%http://tex.stackexchange.com/questions/119905/insert-multiple-figures-in-latex

\usepackage[nomessages]{fp}    %mjm   needed for chemfig and mol2chemfig computed angles  
\usepackage{siunitx}        %mjm  appendix table subsections  
%\usepackage{morefloats}        %mjm   saving up figs for the end   
% fking incompatibvle fuxk ing floatrow 
%\usepackage{float}        %mjm  appendix table subsections  
\usepackage{pbox}        %mjm box off junk  
\usepackage{comment}        %mjm  see the build options  
\usepackage{framed}        %mjm box off junk  
\usepackage{lgrind}        % convert program listings to a form includable in a LaTeX document
% comment out for biblatex test
\usepackage{chapterbib}    % allows a bibliography for each chapter (each labguide has it's own)
%\usepackage{biblatex}   
\usepackage{color}         % produces boxes or entire pages with colored backgrounds
\usepackage{graphics}      % standard graphics specifications
\usepackage[pdftex]{graphicx}      % alternative graphics specifications
%\usepackage{graphicx}      % alternative graphics specifications
\usepackage{longtable}     % helps with long table options
\usepackage{epsf}          % old package handles encapsulated post script issues
\usepackage{bm}            % special 'bold-math' package
\usepackage{url}            % path for stupid jobname   
%\usepackage{asymptote}     % For typesetting of mathematical illustrations
\usepackage{thumbpdf}
\usepackage[colorlinks=true]{hyperref}  % this package should be added after all others
%\usepackage[draft=false, x-bib-pages=\input{\mjmbasename.last_page}, colorlinks=true]{hyperref}  % this package should be added after all others
%\usepackage[draft=false, x-bib-pages=\input{allbib.last_page}, colorlinks=true]{hyperref}  % this package should be added after all others
                                        % use as follows: \url{http://web.mit.edu/8.13}

%http://tex.stackexchange.com/questions/12676/add-notes-under-the-table
%\usepackage{booktabs,caption,fixltx2e}
% not work with subfig????
\usepackage[CaptionAfterwards]{fltpage}
\usepackage{lipsum}

% this does not  work .... 
%\usepackage[utf8]{inputenc}
%\usepackage{tabulary}
\usepackage[para,online,flushleft]{threeparttable}
%\usepackage{threeparttable}

%\usepackage{chemfig}
\usepackage[version=3]{mhchem}        %  
\usepackage{mol2chemfig}
% chemfig vriables maybe
\usepackage{xstring}        %mjm  appendix table subsections  
%\usepackage{chemformula}
% \usepackage{chemmacros}
\usepackage{floatrow}
\usepackage{fancyhdr}
%  underscore in jobmane f 
% https://latex.org/forum/viewtopic.php?t=2975
% this also messes up pdftotext  
%\usepackage[T1]{fontenc}
\usepackage{dcolumn} % https://tex.stackexchange.com/questions/2746/aligning-numbers-by-decimal-points-in-table-columns

\usepackage{catchfile} % mjmaddbib needs to read page count file
\pagestyle{fancy}




\newcolumntype{.}[1]{D{.}{.}{#1}}
%\usepackage[maxfloats=30]{morefloats}   %mjm   saving up figs for the end   
% no param on old version stuck at 36
\usepackage{morefloats}        %mjm   saving up figs for the end   
\usepackage{graphicx}        %mjm   saving up figs for the end   

% will need modificaitons 

% 2020-10-18 extract some new boilerplate 

% https://tex.stackexchange.com/questions/121601/automatically-wrap-the-text-in-verbatim
\usepackage{listings}
\lstset{
basicstyle=\small\ttfamily,
columns=flexible,
breaklines=true
}
%%%%%%%%%%%%%%%%%%%%%%%%%%%%% utilitites

\newcommand{\mjmblackbox}[2]{
 \fbox{
% thi does not ing work right 
\begin{minipage}[t]{\textwidth}
{ \centering{\bf{#1 : }} }
\par
#2
\end{minipage}
}
}
\newcommand{\mjmblackboxno}[2]{
 \fbox{
% thi does not ing work right 
\begin{minipage}[t]{\textwidth}
{ \centering{\bf{#1 : }} }
#2
\end{minipage}
}
}





%%%%%%%%%%%%%%%%%%%%%%%% biblio stuff



\newcommand{\checkrel}[1]{%
  \ifcsname#1\endcsname%
\newcommand{\mjmstatus}{  public NOTES }
\newcommand{\mjmversion}{\mjmrelease}
  \else%
\newcommand{\mjmstatus}{ NOT public NOTES }
\newcommand{\mjmversion}{0.00}
  \fi%
}




% https://tex.stackexchange.com/questions/18089/are-there-any-command-for-producing-the-bibtex-logo
%\def\BibTeX{{\rm B\kern-.05em{\sc i\kern-.025em b}\kern-.08em
%    T\kern-.1667em\lower.7ex\hbox{E}\kern-.125emX}}

\newcommand{\biblogo}{
{{\rm B\kern-.05em{\sc i\kern-.025em b}\kern-.08em
    T\kern-.1667em\lower.7ex\hbox{E}\kern-.125emX}}
{ }  }
%\newcommand{\biblogo}{ Bibte{\it X}  { }  }
\newcommand{\latexlogo}{ \LaTeX  { }  }
\newcommand{\bomtexlogo}{ BomTe{\it X}  { }  }






\newcommand{\mjmvirus}{SARS-Cov-2 }
\newcommand{\mjmdisease}{covid-19 }
\newcommand{\Mjmdisease}{Covid-19 }
\newcommand{\mjmlogo}{ MUQED { }  }
\newcommand{\mjmlinkedin}{ {\bf LinkedIn} { }  }








% https://tex.stackexchange.com/questions/121601/automatically-wrap-the-text-in-verbatim
\usepackage{listings}
\lstset{
basicstyle=\small\ttfamily,
columns=flexible,
breaklines=true
}

% none of this fking fking works for a f F 
\newcommand{\mjmverbatim}{lstlisting}
\newcommand{\mjmbeginverbatim}{\begin{lstlisting}}
\newcommand{\mjmendverbatim}{\end{lstlisting}}

\newcommand{\mjmmangle}[1]{keep/#1}



%# CHANGE VERSION AND STATUS MANUALLY 
% need a draft/notes/release flag

% https://tex.stackexchange.com/questions/5894/latex-conditional-expression
%At the command-line, you can do \def\MYFLAG{} and then test if \MYFLAG is defined in your document (or an included style file) with \ifdefined\MYFLAG ... \else ... \fi.
% needs trailing space for the sample bibtex doh
% leading spaces mess up the entry thought 
\def\xxmjmrelease{0.10 }
\ifdefined\mjmrelease
\newcommand{\mjmstatus}{ PUBLIC NOTES }
\newcommand{\mjmversion}{\mjmrelease} %%%%%%%%%%%%%
\newcommand{\mjmtrno}{MJM-2023-008}
\newcommand{\mjmbib}{\mjmtrno-\mjmversion}
\newcommand{\mjmstatuswarn}{{\bf{  }}   }
% 2021-09-29 wanted version wth for brownie
%\newcommand{\mjmbib}{\mjmtrno-\mjmversion-\mjmrelease}
%\newcommand{\mjmbib}{\mjmtrno}
\else
\newcommand{\mjmstatus}{ NOT public NOTES }
\newcommand{\mjmversion}{0.00} %%%%%%%%%%%%%
\newcommand{\mjmtrno}{MJM-2023-008}
%\newcommand{\mjmbib}{\mjmtrno-\mjmversion}
\newcommand{\mjmbib}{\mjmtrno}
%\newcommand{\mjmstatuswarn}{  }
\newcommand{\mjmstatuswarn}{{\bf{This document is a non-public DRAFT and contents may be speculative or undocumented or simple musings and should be read as such.  }}   }
\fi

%\newcommand{\mjmstatus}{ NOT public NOTES }



%\newcommand{\mjmtitle}{Potential Uses of Microbiome as Nutrient Diagnostic in Unresolved Diseases.}
%\newcommand{\mjmtitle}{Tryptophan as a Unifying Component of Age Associated Diseases.}
\newcommand{\mjmtitle}{Brain Microbiome : Make Your Garden Grow? Feed Your Head }

\ifdefined\grad
\else
\newcommand{\grad}{\nabla}
\fi
\newcommand{\laplace}{\nabla^{2}}
\newcommand{\mjmdx}[2]{\left(\frac{\partial #1 }{\partial #2} \right) }
\newcommand{\mjmdxop}[2]{\frac{\partial  }{\partial #2}\left( #1 \right) }
\newcommand{\mjmdxdx}[2]{\left(\frac{\partial^{2} #1 }{\partial #2^{2}} \right) }
\newcommand{\mjmdxo}[2]{\frac{\partial #1 }{\partial #2} }
\newcommand{\mjmdxx}[2]{\left(\frac{\partial^{2} #1 }{\partial {#2}^{2}} \right) }
\newcommand{\mjmdxy}[3]{\left(\frac{\partial^{2} #1 }{\partial {#2}\partial{#3}} \right) }
\newcommand{\mjmdxyn}[5]{\left(\frac{\partial^{#1} }{\partial^{#2} {#3}\partial^{#4}{#5}} \right) }
\newcommand{\mjmdsq}[2]{\left(\frac{\partial #1 }{\partial #2} \right) ^{2}}
\newcommand{\mjmupdated}[2]{\p  Updated on $#1$ from source #2 \p }
% see tug mail archives, based on discussion 
%\bool_new:N \l_tmpa_bool
\newif\ifbibstarted
\newif\ifbibnamed
\bibstartedfalse
\bibnamedfalse
\newcommand{\mjmtotalbib}{}
%\def\foo{}
\newcommand{\mjmsummabib}[2]{
%\renewcommand{\mjmtotalbib}{ \mjmtotalbib, #1 = #2 }
%\let\foo{ #1 = #2}
}
%%%%%%%%%%%%%%%%%%%%%%%%%%%%%%%%%%%%%%%%%%%%%%%%%%%%%%%%%%%%%%%%

\iffalse
@software{,
  author = {Michael J Marchywka},
  city = {Jasper GA 30143 USA},
  title = { A one-file library for adding machine and human readable bibtex to an article },
abstract={ A simple include file to make bibtex available 
in a document in both machine and human readable format. 
Machine readable is added to extended information 
which can be read with tools such as exiftool ( https://exiftool.org/ ).
While typically not complete at time of pdf creation,
other tools such as toobib can be used to complete the citation
and of course publishers may be able to modify it too as 
more is known. The human readable form need not beincluded
in document types not suited for that but then automated citation
may still be easy.  
See also some exchanges on the Texhax mailing list @tug.org },
institution={},
license={Knowledge sir should be free to all },
publisher={Mike Marchywka},
email={marchywka@hotmail.com},
authorid={orcid.org/0000-0001-9237-455X},
  filename = {mjmaddbib.tex},
  url = {},
  version = {0.0.0},
  date-started = {}
}

<one line to give the program's name and a brief idea of what it does.>


Conceived and written by Mike Marchywka from 2019 to present.
See dates in individual code pieces as they were 
generated from my wizards. 
Copyright (C) <year> <name of author>


This program is free software: you can redistribute it and/or modify it under
the terms of the GNU General Public License as published by the Free Software
Foundation, either version 3 of the License, or (at your option) any later
version.

This program is distributed in the hope that it will be useful, but WITHOUT ANY
WARRANTY; without even the implied warranty of  MERCHANTABILITY or FITNESS FOR
A PARTICULAR PURPOSE. See the GNU General Public License for more details.

You should have received a copy of the GNU General Public License along with
this program.  If not, see <http://www.gnu.org/licenses/>.

   THIS SOFTWARE IS PROVIDED BY THE COPYRIGHT HOLDERS AND CONTRIBUTORS
   "AS IS" AND ANY EXPRESS OR IMPLIED WARRANTIES, INCLUDING, BUT NOT
   LIMITED TO, THE IMPLIED WARRANTIES OF MERCHANTABILITY AND FITNESS FOR
   A PARTICULAR PURPOSE ARE DISCLAIMED.  IN NO EVENT SHALL THE COPYRIGHT OWNER OR
   CONTRIBUTORS BE LIABLE FOR ANY DIRECT, INDIRECT, INCIDENTAL, SPECIAL,
   EXEMPLARY, OR CONSEQUENTIAL DAMAGES (INCLUDING, BUT NOT LIMITED TO,
   PROCUREMENT OF SUBSTITUTE GOODS OR SERVICES; LOSS OF USE, DATA, OR
   PROFITS; OR BUSINESS INTERRUPTION) HOWEVER CAUSED AND ON ANY THEORY OF
   LIABILITY, WHETHER IN CONTRACT, STRICT LIABILITY, OR TORT (INCLUDING
   NEGLIGENCE OR OTHERWISE) ARISING IN ANY WAY OUT OF THE USE OF THIS
   SOFTWARE, EVEN IF ADVISED OF THE POSSIBILITY OF SUCH DAMAGE.


\fi

%\usepackage[pdftex]{graphicx}      % alternative graphics specifications
\usepackage{hyperref}      %


\newcommand{\mjmstartbib}[2]
{
\def\mjmbibentry{@#1\{#2}
\def\mjmbiboneentry{@#1\{#2}
\def\mjmbibpre{x-bib{-}}
\def\mjmday{\day}
%\mjmaddbib{run-day}{\mjmday}
%\mjmaddbib{run-month}{\expandafter\month}
%\mjmaddbib{run-year}{\year}
\mjmaddbib{filename}{\jobname}
\mjmaddbib{run-date}{\today}
}
\newcommand{\mjmbibmunge}[1] {x-bib-#1}

\newcommand{\mjmaddbib}[2]
{
{\hypersetup{pdfinfo={{\mjmbibmunge{#1}}={#2}}}}
\edef\mjmbibentry{\unexpanded\expandafter{\mjmbibentry,}

    #1 =\{#2\} }
%\edef\mjmbiboneentry{\unexpanded\expandafter{\mjmbiboneentry,}
\edef\mjmbiboneentry{\unexpanded\expandafter{\mjmbiboneentry,\linebreak}
#1 =\{#2\}
}

} % mjmaddbib


\newcommand{\mjmshowbib}
{

\mjmbibentry

\}
} % mjmshowbib

\newcommand{\mjmshowbibone}
{
\mjmbiboneentry \}
} % mjmshowbibone

\newcommand{\mjmdonebib}
{

{\hypersetup{pdfinfo={\mjmbibmunge{bibtex}={\mjmshowbibone}}}}

} % mjmdonebib


% af 
\newcommand{\mjminputlisting}[2]
{
%\begin{figure}[H]
\lstinputlisting{#1}
%\caption{#2}


#2

%\end{figure}
}

%%%%%%%%%%%%%%%%%%%%%%%%%%%%%%%%%%%%%%%%%%%%%%%%%%%%%%%%%%%%%%%%%%

%\newcommand{\mjmaddbib}[2]{\hypersetup{ pdfinfo={ x-bib-#1 = {#2}}}\mjmsummabib{#1}{#2}}

\newcommand{\mjmaddbibonly}[2]{\hypersetup{ pdfinfo={ x-bib-#1 = {#2}}}}
\newcommand{\mjmaddbibe}[2]{\hypersetup{ pdfinfo= x-bib-#1 = #2}}

\newcommand{\mjmtable}[2]{
\begin{table}[H] \centering
\begin{tabular}{#1}
#2
\end{tabular}
\end{table}

}
% for archiving and file list 
% David Carlisle You can use   \textbf{\detokenize{ pmg_ratios.svg}} \IfFileExists{ pmg_ratios.sv}{yes}{no} to make the tokens safe for typesetting in text mode.


\newcommand{\mjmusesitem}[2]{

%\item {\bf $ #1 $  } : #2  % 1 is #1 and 2 is  #2 xxx 
\item {{\detokenize{ #1 } }} : #2  % 1 is #1 and 2 is  #2 xxx 
\IfFileExists{#1}{}{{\bf not found} }

} % mjmuses

\newcommand{\mjmreleasewarning}
{
{\bf This is a draft and has not been peer reviewed or completely proof 
read but released in some state where it seems worthwhile given 
time or other constraints. Typographical errors are quite likely 
particularly in manually entered numbers. This work may 
include  output from software which has not been fully debugged.  
For information only, not for use for any particular purpose see
fuller disclaimers in the text.  Caveat Emptor.}
}

\newcommand{\mjmwarningtoo}
{\bf {This is a draft which may not have been fully proofread and certainly
not peer reviewed. Read the disclaimers and take them seriously.
The reader is assumed familiar with the related literature and
controversial issues. For information and thought only not intended
for any particular purpose. Caveat Emptor  }}

\newcommand{\mjmwarntopic}
{\bf {  This work addresses a controversial topic and likely advances one
or more viewspoints that are not well accepted in an attempt to
resolve confusion.   
The reader is assumed familiar with the related literature and
controversial issues and in any case should seek additional 
input from sources the reader trusts likely with differing opinions. 
For information and thought only not intended
for any particular purpose. Caveat Emptor  }}


\newcommand{\mjmwarnfeed}
{\bf { Note that any item given to a non-human must be checked for safety alone and in combination with other ingredients or medicines  for that animal. Animals including dogs and cats have decreased tolerance for many common ingredients in things meant for human consumption. }}


\newcommand{\mjmwarnme}
{\bf { I am not a veterinarian or a doctor or health care professional
 and this is not particular advice
for any given situation.  Read the disclaimers in the appendicies or text, take them seriously and take prudent steps 
to evaluate this information. 
 }}

\newcommand{\mjmexplainbib}
{{The release may use an experimental bibliography code that
is not designed to achieve a particular format but to
allow multiple links to reference works with modifications
to the query string to allow identification of the citing
work for tracking purposes. This may be useful for a bill-of-materials
and purchases later.
}}








\newcommand{\mjmauthor}{Mike J Marchywka }
\newcommand{\mjmmakedate}{2023-09-17 }
\newcommand{\mjmbasename}{\jobname}
%\newcommand{\mjmaddbio}{mjm_tr,releases}
\newcommand{\mjmaddbio}{mjm_tr,releases,dailysnapx}
%\newcommand{\mjmversion}{0.00}
%\newcommand{\mjmtrno}{MJM-2023-008}
%\newcommand{\mjmbibday}{17}
%\newcommand{\mjmbibmo}{09}
%\newcommand{\mjmbibyear}{2023}

% the build script changes these to creation day doh 
\newcommand{\mjmbibday}{17}
\newcommand{\mjmbibmo}{09}
\newcommand{\mjmbibyear}{2023}


\newcommand{\mjmmakebibday}{\number\day}
\newcommand{\mjmmakebibmo}{\number\month}
\newcommand{\mjmmakebibyear}{\number\year}

\newcommand{\mjmbibtype}{techreport}

\newcommand{\mjmbibname}{marchywka-\mjmbib}
\mjmstartbib{\mjmbibtype}{\mjmbibname}


\newcommand{\mjmemail}{marchywka@hotmail.com}
%\newcommand{\mjmaddr}{306 Charles Cox , Canton GA 30115}
\newcommand{\mjmaddr}{44 Crosscreek Trail, Jasper GA 30143}
\mjmaddbib{title}{\mjmtitle}
\mjmaddbib{author}{\mjmauthor}
\mjmaddbib{type}{\mjmbibtype}
%\mjmaddbib{name}{marchywka-\mjmbib}
\mjmaddbib{name}{\mjmbibname}
\mjmaddbib{number}{\mjmtrno}
\mjmaddbib{version}{\mjmversion}
\mjmaddbib{institution}{not institutionalized, independent }
\mjmaddbib{address}{ \mjmaddr}
\mjmaddbib{date}{\today}
\mjmaddbib{startdate}{\mjmbibyear -\mjmbibmo -\mjmbibday }
%\mjmaddbib{day}{\mjmbibday}
%\mjmaddbib{month}{\mjmbibmo}
%\mjmaddbib{year}{\mjmbibyear}
\mjmaddbib{day}{\mjmmakebibday}
\mjmaddbib{month}{\mjmmakebibmo}
\mjmaddbib{year}{\mjmmakebibyear}

\mjmaddbib{author1email}{\mjmemail}
\mjmaddbib{contact}{\mjmemail}
\mjmaddbib{author1id}{orcid.org/0000-0001-9237-455X}
\CatchFileEdef\mjmpages{\mjmbasename.last_page}{\endlinechar=-1\relax}
% TODO FIXME add this to the skeleton text 
%\mjmaddbib{pages}{ \input{\mjmbasename.last_page}}
\mjmaddbib{pages}{ \mjmpages}
%\mjmaddbib{filename}{\mjmbasename}
%\mjmaddbib{bibtex}{\mjmfullbib}
\mjmdonebib

\lhead{\mjmauthor,  \mjmtrno }


%\lhead{M Marchywka,  \mjmtrno }
%\rhead{ \mjmversion not for public release}
%\rhead{ { \today }  v. \mjmversion for release without review }
%\rhead{ { \today }  v. \mjmversion NOT public DRAFT }
%\rhead{ { \today }  v. \mjmversion { }  NOT public NOTES }
\rhead{ { \today }  v. \mjmversion { }  \mjmstatus }

\newfloatcommand{capbtabbox}{table}[][\FBwidth]


%%%% build flags 
%\newlength{\desttabw}  \setlength{\desttabw}{4in}
\newlength{\desttabw}  \setlength{\desttabw}{\textwidth}
\newlength{\chainwidth}  \setlength{\chainwidth}{.4\textwidth }
\newlength{\slantwidth}  \setlength{\slantwidth}{.2\textwidth }
\newlength{\subfigwidth}  \setlength{\subfigwidth}{.3\textwidth }
\newlength{\fullfigwidth}  \setlength{\fullfigwidth}{.8\textwidth }
\newlength{\subwfigwidth}  \setlength{\subwfigwidth}{.75\textwidth }
\newlength{\subwfigwidthrot}  \setlength{\subwfigwidthrot}{\textwidth }
\newlength{\myboxwidth}  \setlength{\myboxwidth}{.3\textwidth }
\newlength{\picwidth}  \setlength{\picwidth}{.4\textwidth }
% set to center for nowmal output 
\newcommand{\destflushtab}{flushleft}
\includecomment{mdpicomment}
\excludecomment{draftcomment}
\excludecomment{badmathcomment}
\excludecomment{showworkcomment}
% this does not fing work ... 

\newcommand{\mjmed}[1]{
%\begin{mjmedx} 
[ mjm : #1   ]
%\end{mjmedx} 
}  

% thinking outload
\newcommand{\mjmtolx}[1]{}
\newcommand{\mjmtolxx}[1]{}
\newcommand{\mjmtol}[1]{
 \fbox{  
% thi does not ing work right 
\begin{minipage}[t]{\textwidth}
{ \centering{\bf{Thinking outloud}} }
\par   
#1 
\end{minipage} 
}
}

\newcommand{\mjmpicture}[3]
{
\begin{figure}[H]
{ \includegraphics[height=3in,width=4in]{keep/#1} }
\caption{#2}
\label{fig:#3}
\end{figure}
} % mjmpicture


\newcommand{\mjmaside}[1]{
 \fbox{  
% thi does not ing work right 
\begin{minipage}[t]{\textwidth}
{ \centering{\bf{Aside: }} }
\par   
#1 
\end{minipage} 
}
}




\newcommand{\mjmgraphics}[1]{#1 }
\newcommand{\mjmfullplot}[1]{\includegraphics[width=\fullfigwidth]{#1}}
%\newcommand{\mjmincludeplot}[1]{\includegraphics[width=\fullfigwidth]{#1}}
%\newcommand{\mjmincludeplot}[1]{\includegraphics[width=\subfigwidth]{#1}}
\newcommand{\mjmincludeplot}[1]{\includegraphics[height=3in,width=\fullfigwidth]{#1}}

% include here as likely to be doc specific 
%\newcommand{\mjmreffig}[1]{Fig. \ref{#1}}
\newcommand{\mjmreffig}[1]{Fig. \ref{fig:#1}}
\newcommand{\mjmreftab}[1]{Table  \ref{tab:#1}}
\newcommand{\mjmrefapp}[1]{Appendix   \ref{appendix:#1}}


%cp yyy2.pdf ~/d/latex/keep/pp20171124biotin.pdf
\newcommand{\mjmdatedplot}[1] 
{ \includegraphics[height=3in,width=\fullfigwidth]{keep/pp20171124#1} }
% right now there are too many figs to save for the end apaprently
% hard limit is 36
\newcommand{\mjmbeginfigure}{\begin{figure}[H] }
%\newcommand{\mjmbeginfigure}{\begin{figure}[p] }
\newcommand{\mjmfigure}[1]{
%\begin{figure}[H]
\mjmbeginfigure

#1 

\end{figure}
}
%\extrafloats{100}

\newcommand{\mjmlisting}[1]
{
\begin{lstlisting} 
#1 
\end{lstlisting}
}

\newcommand{\mjmold}[1]{ } 



\newcommand{\mjmeqn}[1]{\begin{equation} #1 \end{equation}  } 





\begin{document}

\title{\mjmtitle}
\author         {Mike Marchywka}
\email          {\mjmemail}
\thanks{ to cite  or credit this work, see bibtex in \ref{appendix:citing} } 
\date{\today}
\affiliation{\mjmaddr}

\mjmblackboxno{Release Notes  xxxx-xx-xx : }{
This is largely an alternative analysis of
a simple differential abundance measure
between two groups of brains from one recent
publication
\cite{10.3389/fcimb.2023.1123228}
but it points to ambiguity
inherent in a lot of data and puts forth a set of realistic
considerations for alternatives. 
The title derives from literature on symbiosis with plants
related to a surprising number of the organisms considered.
Although "Feed your head" may be importan too :) 
As soon as I can figure out how to download the Bioproject data
I hope to look at sequence level analysis. This may be  another
case of the mixed taxonomy getting in the way of seeing
what is really there.  A good database on metabolism
may be helpful too.  

\mjmreleasewarning

\mjmstatuswarn

\mjmexplainbib

%\mjmwarnfeed

\mjmwarnme

\mjmwarntopic

\bf{ Readers may experience disclaimer fatigue. Doe not proceed if you
are weary or unable to think clearly}


}
\begin{abstract}
A recent work comparing the the post-mortem brain microbiome
of those who died with Alzheimer's to controls 
\cite{10.3389/fcimb.2023.1123228}
provides  excellent differential mesurement of the two
states.
The authors mostly interpreted results
to support a causal role for the associated 
organisms in disease progression
.
They found Staphylococcus epidermidis, among others,  to be more
common in AD brains and this known pathogen may be
one of a group of causative or contributory agents 
yet to be characterized. 
Their results  also suggest the diseased brain has
different  
nutritional or metabolic status .
The AD brains had lower abundances of   organisms 
prone to overgrowth
in benzoate rich conditions ( Pseudomonas )
while relative abundance gains were observed for  organisms 
thought to benefit from low Trp or 5HT (Cutibacterium ) 
and higher iron ( Acidovorax )  and methanol ( Methylbacterium ) . 
This work's title derives from the surprising literature linking
many of these organisms to plant nutrient uptake which may or may not
be relevant in the brain. 
There is also the 
possibility the AD brains lost  beneficial organisms acquired
as early as  conception or the  peri-natal period as 
organisms  enhancing nutrient uptake in plants 
(such as Acinetobacter junii ) 
and  dosed during fetal/peri-natal period during conception,
delivery, or breast feeding were  less abundant.

The possibility of symbiotics  suggests caution in antibiotic
usage while the growth of iron and methanol responsive organisms
helps validate their results as being consistent with
known properties of the disease state such as microbleed and
endogenous methanol disturbances. Interestingly, if
methanol is the chief driver that may still be of microbial
origin but possible from the GI tract.  

The nutrients highlighted by these organisms tend to
support earlier interest in neurotransmitter precursor
amino acids as well as vitamin K. The suggestions of nutrient
uptake enhancing bacteria points to the need however
to work with non-optimal  physical properties such as solubility.
The prominence of methanol organisms may also point
to SAM and methionine which I had not considered before. 

My earlier work appears on the right track although histidine
is not implicated as an issue while endogenous 
methanol production may be important. 
If the organisms in the healthy brain are truly symbiotic,
the notion of absorption aids may have merit in general
and in the GI tract ( surfactants added to diet ). 
One simple intervention to explore is increased stomach
acidity, not equivalent to  abstaining from PPI's, 
 as a way to improve nutrient uptake and minimize methanol production. 

\end{abstract}





\maketitle
\tableofcontents
\newpage

\section{Introduction  }

Despite many interesting discoveries in the lab and clinic,
cause and effect in many biological settings remains difficult
to determine  stifling the design of therapeutic interventions.
Alzheimer's Disease is one important unresolved medical issue
which may exemplify this limitation. For decades, amyloid beta and tau
had both been considered as causal in the sense that removing
either one would reduce disease processes and produce
significant clinical benefits or even a cure
( see citations in any of the AD citations below ).  
That state of affairs is well documented 
in the works
that hint at it unravelling\cite{PMID37833948}  such as a 2002 work 
suggesting that  "tauists" and "baptists" could 'shake hands"
and look for other causes \cite{PMID11801334}. Efforts
continued and recently
Aducanumab was approved despite trials having been stopped for futility
\cite{PMC8491638}
followed by Lecanemab \cite{PMC10119064}
\cite{vanDyck_Swanson_Aisen_Lecanemab_Early_Alzheimer_2023}. 
Other approaches   target  cholinesterase  
 \cite{PMC4052996} 
due to decreased acetylcholine and NMDA receptors 
\cite{PMC6375899} to control  glutamate and Ca 
with limited success.

Recently, the infectious disease hypothesis has gained  
attention \cite{10.1371/journal.ppat.1010929}.
\cite{10.3389/fcimb.2023.1123228}
\cite{Lathe_Schultek_Balin_Establishment_consensus_2023}.
This is motivated by a variety of observations
including  the realization that amyloid
beta is protective and appears as a CNS specific immune response.
A 2022 work concluded AD is an autoimmune disease modualted 
by Trp metabolites  
\cite{DiezCecilia_Kolaj_Santos_Alzheimer_disease_2022}.
However, autoimmune disease always invites the "undetected
pathogen" concern and hence an interest in infectious etiology.
Immune activation and Trp depletion associated with AD were recognized
as early as 2000 \cite{Widner2000}.
Similarly, Parkinson's has been considered as autoimmune with 
possible undetected infection \cite{PMID37741513}.
Perhaps most interesting is the similarities between
various known spirochete infections including Treponema pallidum  and AD
\cite{PMC4399390} especially as  syphilis cases
continue to escalate although in younger high-risk fringe
groups
\cite{Amerson_CastilloValladares_Leslie_Resurgence_Syphilis_2022}
distinct from the typical AD patients.  


Other work has continued on age related cognitive decline in general.
Some of this involves microbes in various ways such as the
GI microbiome and their metabolites.  A mouse model of
questionable relationship to real human diseases for example
was investigated for behavioral issues related to metabolic
and gut microbia properties \cite{PMID37717663}.
In general tryptophan metabolism is now being explored as
a mediator from GI microbiome to brain function
\cite{PMC7231603} with particular interest in 5HT
\cite{PMID25078296}
Some work appears to just emphasize the role of Trp
metabolite and various details
\cite{Savonije_Weaver_Role_Tryptophan_2023}
without considering the likely context pointing to deficiency.

Significant literature exist on gut bacteria or organisms
not within the CNS and their impact on the brain by various
means such as metabolite generation or nutrient modifications.
Metabolites may also exist in ingested food independent of host
microbiome. One interesting work from 2007 identified tryptamine
as a cause of neurodegeneration through Trp-tRNA synthetase 
and the effects may not be overcome with excess Trp
\cite{PMID17114825}. Recent literature on microbially derived
tryptamine and indeed tRNA related signalling and 
particularly Trp-tRNA- related signalling has added details but not
resolved issues completely. 

A nutritional or dietary component has been considered and various
interventions proposed.
Not surprisingly given the observations of decreased
acetylcholine, dietary choline is a popular choice
\cite{PMC7041773}. 
In the case of choline deficiency, given that gut organisms
are thought modulate choline availability to the host
\cite{MartinJBlaser_Intestinal_Microbiota_Composition_Modulates_2015}
, nutrient and neurotransmitter deficiency could be mediated by
infection with the wrong microbes and age dependence due
to impairment of GI tract. In this case eradication of the
offending metabolic routes may be therapeutic but ultimately
the presumed vitamin deficiency still needs to be corrected
to try to repair the GI tract if possible.  



In any case, microbe exposure is not as  age specific
as Alzheimer's Disease and an infectious etiology 
 suggests a good intervention
will be host directed attempting to restore the successful
"youthful' response. 
There are special events such as conception at which
time exposure and response to microbes is unique and difficult
to replicate later in life.
By finding  associations between groups of organisms
and clinical trajectory, work on the infectious hypothesis
will greatly aid evolving hypoteses that consider the
microbial environment such as barrier function and nutrient flows. 

%%%%%%%%%%%%%%%%%%%%%%%%%%%%%%%%%%%%%%%%%%%%%%%%%%%%%%%%%%%%%%%%

The present work builds on my earlier efforts 
that suggest other neurotransmitters, notably
those derived from the "WHY" ( tryptophan histidine tyrosine ) 
amino acids, possibly along with
vitamin K, are more important although a complete solution
will address the age related bottlenecks. 

Herein, I describe how common sources
like lecithin may also promote absorption of other nutrients
making success anecdotes ambiguous.  
One work suggesting amino acid and tryptophan restriction
to counter aging suggests that can impair cognition
\cite{CANFIELD201970}.

The present work seeks to support a nutritional hypothesis,
without or without microbial involvement,  that
cognitive decline and correlates of aging in general are due
to effective starvation for various nutrients in compartment
specific ways. Part of this was alluded to in a response to
\mjmdisease \cite{mmarchywka-MJM-2020-002-0.10}
\cite{mmarchywka-MJM-2020-002-0.12rg}. 
Some interest has recently been expressed in age
related GI damage suggesting interferon-gamma per se
may be part of a problem 
\cite{Omrani_Krepelova_Rasa_IFNupgamma_Stat1_axis_drives_2023}.

Seemingly unrelated work can be linked to nutritional status.
For example, a  causal role for lncRNA such as MEG3
in mediating neuron death in AD has been described
\cite{Balusu_Horre_Thrupp_MEG3_activates_necroptosis_2023}.
You may recall this had previously been implicated
in apoptosis and endothelial cell dysfunction 
related to high glucoase \cite{PMC6958101}
\cite{Tong_Peng_Gu_LncRNA_MEG3_alleviates_high_2019}
as well as pathogen 
\cite{Pawar_Hanisch_PalmaVera_Down_regulated_lncRNA_MEG3_2016}
and tumor control \cite{YAN2023109340} through mTOR. 
With little direct testing of nutritional interventions,
it is interesting that 
 dietary tryptophan was found to improve  a glucose issue 
in fish via lncRNA \cite{JIA2021737256}.
Its also worth noting that in the past  coding regions 
had been given the most
attention while the more idiosyncratic regulatory issues
had been ignored. Control and feedback may be important responses
to various stimuli including nutrition but much analysis 
assumes monotonic static correlations based on protein functions. 
Another line of attack is recent work with aquaporin 4 pointing 
to read through\cite{PMC10566184}.
There is some indication linking Trp ( and Tyr)  or its tRNA 
to read through rates \cite{PMC8136774}.



The work will  interpret the microbiome results
from figure 4 in \cite{10.3389/fcimb.2023.1123228} 
in terms of the literature on the most differentially abundant
organisms ( those at either end of the curve)
which have been listed in \mjmreftab{otuextremes}.
This tends to confirm that their results have some validity
as some organisms suggest the observed changes in iron and
methanol distribution likely in diseased brain while
others relate to differential abundance of nutrients
thought to be neurotransmitter precursors such as
tryptophan and tyrosine. Further consideration of methanol
however adds concern for  two possible endogenous sources:
GI bacteria and excess methyl donors. 
Increasing stomach acidity may reduce GI methanol
production and that is discussed as a low
risk approach to explore although contrary to
many notions that encourage PPI usage. 
Creation from methyl donors
motivates a larger interest in controlling nutrients like 
methionine and supplements like SAM.  
Further consideration  of these organisms in this  
suggests that some of the differences may be due to loss of
symbionts that could have been present since conception
and this hypothesis appears to be novel for AD.
If they were seeded during conception it may be difficult
to re-establish in old age and the use antibiotics would
need to be considered carefully. 
% pursue this in passing as immigration rates are important as is location
%Large community changes also may require reconsideration of BBB permeability in a healthy state.  


\begin{comment}
ir utility
for measuring properties of the CNS  environment such as nutritional
status with only incidental contributions to disease progression.
The work suggests the evidence largely supports an infectious
etiology but I try to show here that in fact the microbes
may be making an exceelent relevant measurement of nutrient
status in the brain. Some of the properties of the differentially
abundant organisms however do suggest way in which they may
assist or impair brain function by their presence.  
The authors produced a minimally processed list of OTU's that
differed in some ways between disease and control samples
and they tabulated the most extreme differences in the text
which is reproduced below, \mjmreftab{otuextremes}.

%Other views of genetics have suggested that in essence amyloid
%beta mediates effects through iron and perceived hypoxia
%\cite{Lardelli_Alternative_View_2023} or in essence nutritional
%issues. 
\end{comment}

\section{The Extreme Differentially Abundant OTU's  }

\mjmreftab{otuextremes} lists
the organisms implied to be
associated with health or AD as  taken from figure 4 in
 \cite{10.3389/fcimb.2023.1123228} . 
they  are discussed in an order related to their  
features. Some have been ignored due to lack 
of interesting literature although some inferences may
be able to be made later by appeal to the 16s sequence later.
Host-organism relationships can be quite diverse 
and many organisms tend to be benign or beneficial to humans
but are also found to become pathogenic under conditions
that may not be well defined. 


\begin{table}[H] \centering
\begin{tabular}{|l|r|c|l|}
\hline
\multicolumn{4}{|c|}{Title}\\
\hline
\multicolumn{4}{|c|}{ Six OTUs are more abundant in the control group:  } \\
\hline
Acinetobacter junii &&& common in soil animals human water soil, plant growth promoter, from birth? \\
Comamonas jiangduensis&&& plant symbiont, genus uses Trp  \\
Cloacibacterium normanense&&& breast milk aromatic degrading \\
Pseudomonas putida,&&& denitrifyin,benzoate overgrowth  breast milk  \\
Pseudomonas thermotolerans&&& may not use benzoate \\
Diaphorobacter nitroreducens&&& sludge not benzoate, methanol,  or sugars tetracycline favored  \\
\hline
\multicolumn{4}{|c|}{ Seven OTUs are more abundant in the AD group:  } \\
\hline
Cutibacterium acnes &&& skin resident, overgrowth in 5HT depletion  \\
Staphylococcus epidermidis &&& skin resident, known pathogen in infant brains \\
Acidovorax ebreus &&& Fe oxidation \\
Acinetobacter tjernbergiae &&& sludge uses histidine but not others  \\
Acidovorax temperans&&& Fe oxidation \\
(Novi)herbaspirillum soli &&& volcanic burned soil plant enhancer  \\
Methylobacterium goesingense&&& methanol consumer \\
\hline
\end{tabular}
\caption{ Miscellaneous features of 
OTU's identified in \cite{10.3389/fcimb.2023.1123228} 
that are more or less common in AD vs control post mortem brain
sections after minimal processing. In order of "Extremism"
with first entry being the most extreme in each category. 
}
\label{tab:otuextremes}
\end{table}

\newcommand{\mjmotu}[1]{ {\bf #1 :  } }
\newcommand{\mjmotuc}[1]{ {\it #1 :  } }
\newcommand{\diaphorobacter}{ % nitroreducens 
\mjmotu{Diaphorobacter}
\mjmotuc{Increased in healthy}
Diaphorobacter nitroreducens was first isolated
from activated sludge in 2002 and found to use polyhydroxybutyrate
but no common sugars  and notably not used were, 
" methanol, caprylate, citrate, benzoate, serine, and histidine"
\cite{Khan_Hiraishi_Diaphorobacter_nitroreducens_2002}.

Diaphorobacter along with Pseudomonas increased in abudance
with addition of tetracycline into a nitrogen reducing 
reactor with mixed bacteria culture \cite{XU2022113652}. 

} %Diaphorobacter nitroreducens 





%Staphylococcus epidermidis 
\newcommand{\staphylococcus}{
\mjmotu{Staphylococcus}
\mjmotuc{Increased in AD}
Until recently Staphylococcus epidermidis was not
considered pathogenic as a normal inhabitant of the skin but
is now recognized as a common pathogen \cite{PMC2807625}.
If there is a specific pathogen responsible for age associated
cognition defects this is the most suspicious of the group.
It is the most common infection in pre-term infants and
activates microglia and modulates BBB permeability
\cite{Gravina_Ardalan_Chumak_Transcriptome_network_analysis_links_2023}.
Like some of the other bacteria here, it has been isolated from  planted
areas and in particular apple 
orchards and found to promote the growth of clover \cite{Cheng2021}.
One report suggested it is a symbiont in the female mouse reproductive
tract \cite{ONO201511}.
It was found in higher amounts in infant fecal micrbiome when 
vitamin K deficient \cite{Benno_Sawada_Mitsuoka_Intestinal_Microflora_1985}
which may be expected as "leakage" or diffusion from the skin and reduced
coagulation barrier function. 

As a  coagulage-negative Staphylococcus, it can secrete
antibacterials such as 6-thioguanine a purine analog
inhibiting purine synthesis in susceptible organisms
\cite{Chin_Goncheva_Flannagan_Coagulase_negative_staphylococci_release_2021}.
It would be interesting to check community for overall entropy
and susceptibility to see if this is a relevant factor. 
In particular, determine  if A junii is susceptible. 

Trp metabolite have some activity against it 
\cite{Narui_Noguchi_Saito_Anti_infectious_Activity_2009}
suggesting its presence may indicate low Trp metabolism due
to lack of tryptophan or reduced activity of any surviving 
brain cells. 


}%Staphylococcus epidermidis 


%Noviherbaspirillum soli &&& \\
\newcommand{\noviherbaspirillum}{
\mjmotu{(Novi)herbaspirillum}
\mjmotuc{Increased in AD}
Originally isolated from plants in volcanic ash, 
and assigned to Herbaspirillum,
Noviherbaspirillum soli  is thought to be siderophore producing
and plant growth promoting and  will assimilate 
lactate and hydroxybenzoate
\cite{Carro_Rivas_LeonBarrios_Herbaspirillum_canariense_2012}
possibly making it parasitic if it predominantly
uses lactate in the sick brain.
The distinction between "parasite competing for scarce lactate"
and " symbiont benefiting from excess benzoate" is 
discussed later as there is a lot of ambiguity in all cases  without
networked rate equations.
It is also possible that more lactate is available due to
astrocyte production in excess of demand by dysregulated neurons.
H soli changed genus circa 2014 from Herbispirillum 
after proposal in 2013
\cite{Lin_Hameed_Arun_Description_Noviherbaspirillum_malthae_2013}.
%Found with family in diesel wastes etc. 
A relative, 
N. denitrification HC18,  can methylate methyl-$As^{3}$ but
not $As^{3}$ \cite{PMC8881391}.
Another relative,  Herbaspirillum sp. WT00C,
 is an endophytic  selenite reducer specific to tea-plants \cite{PMC7075828}.
Noviherbispirillum  was one of a few genera to increase in abundance
in a soil sample with the addition of pyrogenic organic matter
\cite{Zeba_Berry_Fischer_Soil_carbon_mineralization_2023}

Species of Herbaspirillum  considered plant growth promoting bacteria ( PGPB )
were described as having low siderophore production
but P solubilizing ability as endophytes \cite{PMC4763327}.

} %Noviherbaspirillum soli &&& \\
% Methylobacterium goesingense
\newcommand{\methylbacterium}{
\mjmotu{Methylbacterium}
\mjmotuc{Increased in AD}
Methylobacterium are notable for metabolism of methanol
and succinate in beneficial association with plants 
\cite{ChenyuDu_Consumption_Methanol_2012} .
Indeed, increased formaldehyde is observed in neurological patients
and the elderly
\cite{Dorokhov_Shindyapina_Sheshukova_Metabolic_Methanol_Molecular_Pathways_2015}
suggesting the presence of Methylbacterium is indicative
of methanol accumulation  and perhaps mitigation of  formaldehyde 
damage by oxidizing methanol.
It is converted to formaldehyde by a periplasmic dehydrogenase
and in the cytoplasm converted to CO2 or assimilated
\cite{PMC1287603}.
Associations with plants may be 
endophytic or epiphytic
and appears to be largely due to methanol released during pectin
metabolism
\cite{PMC1287603}.

} % Methylobacterium goesingense



\newcommand{\acidovorax}{
\mjmotu{Acidovorax}
\mjmotuc{Increased in AD, 2 species}


Acidovorax was found in root endosphere of Chinese Chives
\cite{PMC8883035}.


As with some other genera, Acidovorax interactions with plants
are well documented
\cite{SBurdmanfirstRRWalcott_Plant_Pathogenic_Acidovorax_2018}
. Beneficial and pathological relationships
have been distinguished by genomes with beneficial
abilities to sense, transport, and synthesize useful
molecules while pathogens have secretion systems
and ability to use plant synthesized lipids \cite{PMC8767351}
.
 Besides acetate, Acidovorax can use lactate and
citrate as carbon sources  \cite{PMC3233087}.
It is notable for oxidation of Fe-II and encrusting
mineralization
\cite{Chakraborty_Roden_Schieber_Enhanced_Growth_Acidovorax_2011}
. Although the benefits of iron oxidation have been demonstrated,
 encrustation depends on iron concentration and perhaps absence of 
chelators\cite{PMC3233087}.
This may generate a distinctive "green rust" 
\cite{10.3389/fmicb.2019.01494}
\cite{Pantke_Obst_Benzerara_Green_Rust_Formation_during_2012}.
Arguably precipitation is a better fate for excess iron than
some alternatives so even this may be mitigating although
not sustainable as minerals accumulate.
} % acidovorax

\newcommand{\acinetobacter}{
\mjmotu{Acinetobacter}
\mjmotuc{Species fluctuate}
%%%%%%%%%%%%%%%%%%%%%%%%%%%%%%%%%%%%%%%%%%%%%%%%%%%%%%%%%%%%%%%%%
The Acinetobacter distribution appears to be the most confusing
with junii over abundance in the control group and 
Acinetobacter tjernbergiae in the AD brains.
Interestingly some species can live off of tryptophan metabolite kyneurinine and
both of these species lack a putative Mn transporter  
\cite{10.1371/journal.pgen.1010020}
( that work \cite{10.1371/journal.pgen.1010020} also has some
details of genetics among the pathogens ). 
Acinetobacter junii was the organism most overabundant in
the control group. At least in combination with other organisms
appears to promote plant growth \cite{Padmavathi2015}.
Described as "hypertolerant" to metals such as
arsenic \cite{10.1111/jam.14179} and lead \cite{Kushwaha2017}
\cite{PMC6863301} %low levels of similar metals may favor it over others. 
The Acinetobacter in general are described as versatile plant growth
promoters with beneficial effects on availability or buffering of
nutrients like phosphate and metals \cite{MUJUMDAR2023327}.
A junii is rarely a human pathogen but infections have been
documented and more likely in thos with prior antibiotic use, invasive
procedures, or cancer \cite{Clinical_characteristics_patients_Hung_2009}
in preterm infants
\cite{PMC120562}.
Intriguingly, it is much more common in the fecal microbiome of
those delivered by vaginal
birth than cesarean \cite{Pandey2012}
and exposure of cesarean infants to maternal vaginal fluid
improved some aspects of neurodevelopment 
\cite{Zhou_Qiu_Wang_Effects_vaginal_microbiota_2023}.

Pathogenic species of Acinetobacter appear to rely on histidine
\cite{MarvinWhiteley_Histidine_Utilization_2020}
A junii was originally identified in 1985 however by its ability 
to use L-histidine
\cite{Kollimuttathuillam_Bethel_Shaaban_Case_Acinetobacter_2021}
\cite{Bouvet_Grimont_Taxonomy_Genus_1986}.
It is possible that its presence reflect adequate CNS histidine
but refer to \mjmrefapp{interp} for a more complete description.

Some works compare the species in good details such as 
\cite{VanAssche_Assche_}
\cite{Vaneechoutte_Nemec_Kampfer_Acinetobacter_2015}Z
although no obvious differences between A junii and
tjernbergiae were apparent.
} %\acinetobacter

%Acinetobacter tjernbergiae &&& sludge uses histidine but not others  \\
\newcommand{\tjernbergiae} { % &&& sludge uses histidine but not others  \\

Substrate selections for A. tjernbergiae were listed as, 
\cite{Carr_Kampfer_Patel_Seven_novel_species_2003}
\begin{quote}
Using the method of Ka¨mpfer
et al. (1993), L-arginine, L-histidine and quinate are all used
as sole sources of carbon and energy and some strains utilize
DL-aspartate and L-leucinamide. cis-Aconitate, pimelate,
trans-aconitate, adipate, 4-aminobutyrate, azelate, citrate,
glutarate, malonate, oxoisocaprate, suberate, b-alanine,
L-aspartate, L-glutamate, L-leucine, L-phenylalanine,
L-tryptophan, 4-hydroxybenzoate and phenylacetate are not
utilized.
\end{quote}

} %Acinetobacter tjernbergiae 

\newcommand{\putida}{
P. putida is known for flexibility 
\cite{Molina_Rosa_Nogales_Pseudomonas_putida_2019}
\cite{PMC9004399}
suggesting its presence is not particularly informative
without phenotype or mRNA information. 
It can be denitrifying \cite{PMID19111647}.
However, its potential utility in bioproduction has
created a significant amount of literature on it
including glutarate sensing and metabolism
\cite{NinaRSalama_Regulation_Glutarate_Catabolism_2019}.
While glutarate is a toxic product or amino acid metabolism,
it would not be clear if P putida would be more fit in
a high or low concentration setting but may be able to
remove it. 
A rough form has been found in biofilms along with
Acinetobacter species \cite{PMC1913468}.
However, Pseudomonads and other soil dwelling organisms 
are known for degrading benzoate derived from lignin
\cite{PMC5779716} and could also be derived from tyrosine
or similar neurotransmitters.
} % putida


 %/seedling 
%Pseudomonas thermotolerans&&& \\

\newcommand{\thermotolerans}{
Pseudomonas thermotolerans is notable for possibly
not using benzoate \cite{Manaia_Pseudomonas_thermotolerans_2002}.
Although the real issue is aromatic amino acid usage and early
work did suggest a strain specifivity at least with P aeruginosa,
"28 of 29 strains
grew on tyrosine as a sole source of carbon,
whereas only seven grew on phenylalanine as a
sole source of carbon"
\cite{Calhoun_Pierson_Jensen_Channel_Shuttle_Mechanism_1973}.

} %Pseudomonas thermotolerans&&& \\


\newcommand{\cutibacterium}{
\mjmotu{Cutibacterium}
\mjmotuc{Increased in AD}
Prior to 2016, Cutibacterium acnes was in the genus Propionibacterium
with a 2019 proposal for supbspecies defendes added to acnes and elongatum
which is associated with  progressive macular
hypomelanosis  rather than acne
\cite{Dekio_McDowell_Sakamoto_Proposal_combination_2019}. 
If it is elongatum, relationship to a pigmentation disorcer
suggests interaction with tyrosine or tyrosinase.
Treatments of hypomelanosis may include 
benzoyl containing products
\cite{PMC7772758} although transformations remain unknown,
C. acnes is thought to be able to synthesize biotin
as this has been recognized as a skin or hair issue
\cite{GoularteSilva_Paulino_Ketoconazole_beyond_antifungal_activity_2022}
and sometimes comes up in brain health. Presumably
them low biotin would favor it and signal an unhealthy
environment.  


The species is itself quite diverse.
Interestingly, it is credited as a pathogen in  many
infections with a strain-specific antibiotic,
thiopeptide cutimycin, which is supposed to kill S. epidermidis
although the interaction appears to be two-way
\cite{10.3389/fmicb.2021.673845}. An antibiotic  would be surprising given
that S. epidermidis abundance has increased too. 
This could be a candidate for an AD pathogen but given its 
ubiquity it would need other changes in the host.
I very recent work on fungal infection found Cutibacterium 
expansion in Tph-/- mice which were unable to convert tryptophan
into 5HT \cite{Renga_DOnofrio_Pariano_Bridging_host_microbiota_2023}.
Even though this intervention may have raised tryptophan levels,
it acted similarly to tryptophan depletion in that the tryptophan
metabolite 5HT was missing allowing for proliferation. 
"Serotonin degeneration " bas been implicated in cognitive
declines \cite{PMID37718818} and it may not be clear
in which direction cause and effect operate but 
degeneration in response to limited tryptophan
would be quite plausible. 
} % cutibacterium}


\newcommand{\comamonas}{
\mjmotu{Comamonas}
\mjmotuc{Increased in healthy}
The Comamonas genus was also observed to be less abundant
in intestinal microbiome of AD patients
\cite{JianpingJia_Intestinal_Microflora_Changes_2022}
suggesting the brain and intestinal abundances may correlate
although not mentioned in a similar 2017 analysis
\cite{Vogt_Kerby_DillMcFarland_microbiome_alterations_2017}
or a recent review article
\cite{Li_Zhu_Zhang_intestinal_microbiome_2018}.
\mjmtol{check for name changes lol}

Comamonas jiangduensis known for producing biosurfactant in
agricultural soil \cite{Sun_Zhang_Chen_Comamonas_jiangduensis_2013}
They are infrequent human pathogens and some members were originally
classified as Pseudomonas \cite{PMC9504711}.
Comamonas acidovorans, as part of plant rhizobacteria, with tryptophan was observed to promote
lettuce seedling root elongation \cite{Barazani2000} although
10mM tryptophan inhibited its growth. 

Comamonas genus is also considered metal tolerant 
\cite{Caracciolo_Terenzi_Rhizosphere_Microbial_Communities_2021}.
Potentially then low levels of heavy metals could favor
these organisms leading to increases in absolute abundance. 
Depending on the numbers then, it is
possible that some cognitive benefits could result from small
amounts of these toxic elements. While perhaps not
likely, its important to remember issues like this especially
when the obvious interpretations don't lead to useful
interventions or "go over like a lead balloon." 



} % comamonas

\newcommand{\cloacibacterium}{
\mjmotu{Cloacibacterium}
\mjmotuc{Increased in healthy}
Cloacibacterium was found abundant in stem endosphere while
 Chinese Chives
\cite{PMC8883035}.
Isolated from municipal wastewater by 2006
\cite{PMID16738108},
Cloacibacterium normanense is electrochemically active
\cite{Aparna_Meignanalakshmi_Comparison_power_generation_}
\cite{Yee_Deutzmann_Spormann_Cultivating_electroactive_microbesfrom_field_}
Along with P. putida, it is a component of breast milk
(  Guatemalan mothers ) 
 and both
have aromatic hydrocarbon degrading capabilities
\cite{PMC7907006}.
Its route to the milk is controversial however
\cite{10.3389/fmicb.2022.885588}.


Cloacibacterium normanense is 
listed as an endophyte of plants grown in textile waste
water \cite{Shehzadi_Fatima_Imran_Ecology_bacterial_endophytes_2016}.

\mjmtol{check direction and numbers with author, also not right next to Geobacillis not Geobacter lol. }
It was one of the most increased  genera in recurrence free
patients' BAL fluid prior to resection of early stage
non-small cell lung cancer 
\cite{Patnaik_Cortes_Kannisto_Lower_airway_bacterial_microbiome_2021}
.


Cloacibacterium  was detected in an exploration of 
breast microbiome and metabolites in triple negative
breast cancer \cite{Wang_Rong_Zhao_microbial_metabolite_trimethylamine_2022}
but I Was unable to determine from the supplementary
information which samples were "IM." 
\mjmtol{ from their file mmc4, it looks like several p-values are lower than TMAO including lactate, Trp, and Trp derivatives }
The role of TMAO has been an matter of debate and has featured
some of the same issues related to causal role now being
explored for brain microbes and metaboliates \cite{PMC5127123}. 
The genus was also found differentially abundant in certain
types of artertitis along with metabolic changes 
\cite{PMC9224234}. \mjmtol{ The reference also mentions FISH which may be
useful in localization.}

} % \cloacibacterium



\cutibacterium

\staphylococcus

\acidovorax

\noviherbaspirillum

\methylbacterium

\comamonas

\cloacibacterium

\mjmotu{Pseudomonas }
\mjmotuc{Increased in healthy}
\putida

\thermotolerans

\diaphorobacter


\acinetobacter
\tjernbergiae




% spell

\section{ Microbes, Nutrients, and Disease }



Organisms or communities may be considered pathogenic,
beneficial, or neutral, causal or responders 
with these attributes defined by the result of
the subject's removal from the host. 
Novel data such as these abundances also needs to
be "sanity checked" or validated. 
There are recurring issues with this kind of data
even in this short work and some common caveats
are described in \mjmrefapp{interp}.

The original work \cite{10.3389/fcimb.2023.1123228}
 analyzes the data to suggest that  
 a pathogenic organism or community is causing the disease
and it removal or attack can have clinical benefit. 
However, looking at the most differentially abundant
OTU's, along with other known or suspected characteristics
of the disease state,
 it is easy to make a case for pre-existing beneficial organisms
being replaced by disease related emergent organisms 
largely reacting to and mitigating a changed environment. 
The distinction is important because in the former case
broad spectrum antibiotics with good CNS penetration may be
explored but in the latter case restoration of nutrient
flow and targeted eradication or seeding may be more prudent.
The easy reconciliation between observed abundance variation
and disease properties also helps validate the microbiome
results. 
 
Taken together, these  differential abundances
can suggest that
the diseased brain is reduced in the supply
of tryptophan or metabolites, benzoate or parents
like tyrosine, while having an excess of
iron and methanol. 
Reduced levels of vitamin K and biotin may also be factors. 


Most or  all of these genera, notably Methylbacterium, Pseudomonas,
Acinetobacter and Herbaspirillum, have shown some activity as 
plant growth promoting bacteria (PGPB) \cite{GomezGodinez_AguirreNoyola_MartinezRomero_Look_Plant_2023}.
Comamonas jiangduensis, more common in health, 
is known for producing biosurfactant in
agricultural soil \cite{Sun_Zhang_Chen_Comamonas_jiangduensis_2013}
while Noviherbaspirillum soli, more common in AD, is found in 
volcanic soils and may be  an early colonizer presumably
  suited to lower fertility conditions  suggesting in general
the disease brain may be nutrient poor.


\subsection{Tryptophan related}

Excess tryptophan may suppress the skin bacteria,
the C acnes and S epidermidis.
Reduced tryptophan may be due to decreased intake
with age, decreased uptake due to GI or vascular
impairment, or decreased signalling to bring peripheral
Trp to the brain. Metabolite decreases may be purely
due to Trp decrease or decreased functions or
host cells and bacteria that would produce metabolites
such as 5HT. 
Similar considerations apply to all nutrients including benzoate and tyrosine
with all caveats in \mjmrefapp{interp} applying.

Some reference works exist on various aspects of
metabolism and taxonomy. One work on tryptophan
synthesis and IAA production shows for example
the abilities of A. junii and P. putida,
which have IPA pathways,  in the 
context of others
\cite{SuarezPerez_Bacterial_gene_diversity_related_2020}.
while IAA appears be  deleterious at mg/kg for animal 
fetus
\cite{Furukawa_Abe_Usuda_Indole_Acetic_Acid_2004}
\cite{FURUKAWA200743}
and correlates with cognitive decline in CKD patients
\cite{LIN201985},  to the brain,
IPA has been investigated as a therapeutic in phase II
clinical trials \cite{Politi1999} and it other brain
related effects \cite{Moroni1991}\cite{Bartolini2003}.
Note that the above "IPA" is indole pyruvic acid  NOT indole propinoic
acid although IPA is used for the latter in at least one work on intestinal
production of Trp derivatives suggesting both may be beneficial and
that IAA production is known to occur
\cite{PMC8972051}. As early as 1999, indole-3-propionic acid
was investigtaed for effects against AD \cite{Chyan_Poeggeler_Omar_Potent_Neuroprotective_Properties_against_1999}.
It is quite possible that for a beneficial relationship
the organism has to adapt its metabolism to the human host
versus a typical plant.

As early as 2000, it was demonstrated that acute tryptophan
depletion could worsen cognition among AD patients
\cite{Porter_Cognitive_Deficit_Induced_2000} but that does not
prove that the dominent cause of the natural disease could
be corrected by providing more Trp. A 2003 stody of healthy 
and AD patients demonstrated cognitive impairment by
"depleting" trp in a comparison of two amino acid drinks with and
without tryptophan \cite{PORTER_LUNN_OBRIEN_Effects_acute_tryptophan_2003}
suggesting amino acid competition may be significant in the elderly.  

A interesting 2010 study in Drosophila demonstrated complete recovery
from expression of amyloid beta with oral  
1,4-naphthoquinon-2-yl-L-tryptophan which was designed based
on observations of quinones preventing aggregation 
\cite{JulietAnnGerrard_Complete_Phenotypic_Recovery_2010}.
Presumably, vitamin K then could exert similar effects
and the compound itself could be broken down to provide
excess tryptophan of unknown significance. It is not know if
follow up work occured. 
Vitamin K deficiency was also recognized as early as 2001 
as a contribut0or AD and cardiovascular disease 
\cite{ALLISON2001151} with modern work continuing to
assess the situation \cite{10.3389/fneur.2019.00239}.
 


\subsection{Tyrosine related}


Repeated application of benzalkonium chlorides to the Lascaux Cave
created an overgrowth dominated by Ralstonia and Pseudomonas
\cite{Bastian2009}.
This suggests a competitive advantage in aromatic rich settings 
Benzoate consumption has a putative benefit in dementia
\cite{Lin_Chen_Wang_Effect_Sodium_Benzoate_2021}
and it could act to feed symbiotics or aid absorption
of nutrients including tryptophan. 


Studies relatiing P putida and aeruginosa response
to aromtic amino acids as in cystic fibrosid sputum 
have shown some common and species specific features. 
\cite{PMC2876504}.
Swimming and surfactant production as a function of
amino acid exposuire have been investigated at least
for P. aeruginosa \cite{PMC94731}.

\mjmtol{ D-typroinse but only in combination with an antibiotic
incredible synerty in kill rates 
\cite{Jia_Yang_Xu_Mitigation_nitrate_2017}. D-amino acids 
such as D-serine and D-amino acid oxidase are well known in the brain.  }


\subsection{Histidine  related}
The only obvious relationship to histidine sugests it is probably
adequate or more or less conssitent based on Aceinetobacter. 
See howver \mjmrefapp{interp} for general ambiguities.


\subsection{Methanol   related}

The alert to methanol was a surprise but supported
by signbificant literature. 
As early as 2014, exploration of chronic methanol feeding
relationship to AD pathology was explored \cite{PMID24787915}
with feeding to monkeys producing tau phosphorylation and amyloid
plaques \cite{PMID24787917} similar to AD .

Interestinly investigation on APOE-4 and ethanol consumption 
\cite{LI20181} may be consistent with an important role
for methanol in AD pathology. 

 
Supplement usage related to SAM may need to be more carefully
considered
\cite{Fukumoto_Ito_Saer_Excess_adenosylmethionine_inhibits_2022}.
\mjmtol{ this is a duplucat citation used later } 
Some evidence suggests excess methionine may impair memory for example
\cite{TapiaRojas_Lindsay_MontecinosOliva_methionine_2015}.
A link to Parkinson's-like symptoms has already been considered as early as 2010
\cite{PMC2885904} and investigation into supplement usage may
be useful. 

Increased methanol production or concentration anyway is known in old age
\cite{PMID25834233}
 and may point
to better nutrient optimization.


\subsection{Iron and Vitamin K   related}

The abundance of iron and methanol responsive organisms is also
consistent with prior expectations helping to validate
their initial results.  
iron metabolizing organisms would presumably benefit from
heme influx due to vascular issues like CAA or microbleeding
which is common in AD \cite{PMC5144472}. 


Reduced vitamin K may be suspected
due to the iron influx which may be due to angiopathy
and defective coagulation.  Biotin may also
be a factor. 

Vitamin K in aprticular may not be fully appreaciated but
it apparently important for CNS sulbatide regulation
\cite{PMID8914944}
\cite{PMC6110503}
and the sulfatides thsmelves have important but confusing effects
on coagulation \cite{Kyogashima2001}.
There is a tendency to jump to anticoauglants with the
appearance of inappropriate coagulation but it is necessary
to isolate damage and more vitamin K may be indicated
i\cite{marchywka-MJM-2022-015-0.20c}.


Acidovorax iron oxidation  is intriquing as I suspected mineral deposits were part
of "old age patholgoy" but the question remains if the growth
benefit is due to extra iron supply from heme suggesitng
it is just a symptom of a large leakage problem. Initially, 
I would have suspected calcium minerals and a role for vitamin K.
 
\subsection{Exposure, Vulnerability, and Uptake into CNS }


While C. acnes and S. epidermidis could both
be considered possible pathogens, their lifelong
residence on the skin appears difficult to
reconcile with the age distribution of AD.
However, as would be expected with loer quality
barriers due to protein quality correlating
with old age, their ability to colonize
the brain would be approaching that
which existed at birth ( with vitamin K deficiency lol)
 or indeed conception.



A junii original topic here, 
Vaginal microbiome required more study although some
indications are existence of organisms over represented
in healthy brain. 

Intriguingly, it is much more common in the fecal microbiome of
those delivered by vaginal
birth than cesarean \cite{Pandey2012}
and exposure of cesarean infants to maternal vaginal fluid
improved some aspects of neurodevelopment 
\cite{Zhou_Qiu_Wang_Effects_vaginal_microbiota_2023}.



Possibly the semen carries some of the speculated symbiants
\cite{Koort_Sosa_Turk_Lactobacillus_crispatus_2023}
\cite{10.3389/fcimb.2022.815786}
making "infection-at-conception" possible. 
At least one study did find an increased risk of dementia with
antibiotic usage
\cite{10.3389/fphar.2022.888333}
and a very recent review provides more context for the issue
\cite{KarenMOttemann_2023}.

It is possible that it is actually a symviotic organism lost
over the lifetime although that too may be triggered by nutritional
status. 

\section{Discussion  }

\subsection{ Plant Symbionts: Illusion or Nurse Cells?  }


The appearance of plant symbiants in the CNS is
intriguing if accurate although it could simply
be the result of "literature skew" and coincidence.
First, it would be helpful
to determine where they are and how they have been
missed for so long. Possibly many exist as spores or 
other condensed or quiescent forms. Consideration of the
isolation methods may be worthwile. If true, it may 
be purely coincidental but the known structure of the bain
involves compartmentalized metabolism with astrocytes
already acting as "nurse cells" to provide lactate
to neurons for peak energy demands
\cite{PMID12742077}. The existence of prokaryoric nurse
cells would not be unreasonable if they are common in 
the conception environment. Implications for brain
evolution could be significant.  
My earlier work concerntrated on GI impairments since
they are often ignored features of aging. Vascular
issues present another barrier to nutrient exchange with
active organs especially the brain. A third barrier could
be overcome with substances excreted by microbes as
most of the nutrients considered are hydrophobic.  

The status of "immune privilege" \cite{Proulx_Engelhardt_Central_nervous_system_zoning_2022}
 then may be as much
to nurture bacteria as neurons. 


\subsection{Informing Interventions   }


I outlined a lot of unpublished work in my initial
response to \mjmdisease including the
specific problems with low stomach acid
in the background of issues with tryptophan, tyrosine, and
vitamin K 
\cite{mmarchywka-MJM-2020-002-0.10}.

In fact a 2023 work suggested that frailty more
than age per se correlated with \mjmdisease severity
as well as cognitive issues
\cite{Matsumoto_Shibata_Kishi_Long_COVID_hypertension_2023}.
A link between sarcopenia and cognition is suspected
in general at least in some populations
\cite{PMC9965467}.

As low blood levels of Trp may not be observed,
it turned out that there is some indication of
stress transport
\cite{ mmarchywka-MJM-2021-007-.1-table-rg}
thought to be similar to the case with biotin
where the liver imports less during starvation while
brain metabolism is largely preserved
\cite{PachecoAlvarez_SolorzanoVargas_Gravel_Paradoxical_Regulation_Biotin_2004}.

A  2022  literature review tends to support that amino acid
fluxes in sarcopenia reflect low truptophan levels
\cite{10.3389/fendo.2021.725518},
\begin{quote}
However, a study showed that the levels of isoleucine, leucine, tryptophan, serotonin, and methionine in the participants with low muscle quality were significantly higher than that in the participants with high muscle quality, which may be attributed to impaired metabolism of amino acids, resulting in reduced uptake of skeletal muscle, and thus increased circulating plasma amino acid levels (15). Inconsistencies in amino acid profiles in patients with sarcopenia will lead to variations in clinical practice and research.
\end{quote}

In fact, it is known that Trp consumed with BCAA's may not be well transpoted
to the CNS presumably raising blood levels but not brain levels. 

In fruit mashes, fermentation pH can be a large factor in methanol
production as well as pectin content 
\cite{Blumenthal_Steger_Einfalt_Methanol_Mitigation_during_Manufacturing_2021}.
Reduction of acidity from 2.5 to pH 3.5 may almost double
methanol content. 
Interestingly, endogenous production of ethanol from consumed
sugars maybe reduced with citric acid consumption 
\cite{10.3389/fmicb.2016.00047} while no direct experiments
with methanol production have been found in the literature yet.
PPI usage has a controversial relationship to dementia
in general. 
One recent study found a significant increase after years or usage
\cite{Northuis10.1212/WNL.0000000000207747}
and another 2022 study found associations
including a  bias towards APOE4 carriers
\cite{Zhang_Li_Chen_Regular_proton_pump_inhibitor_2022}.
Although some meta-analyzes and reviews hace concluded the association is
not there 
\cite{PMC10229084}
r only a problem in those using two rather than one type of PPI
\cite{TorresBondia_Dakterzada_Galvan_Proton_pump_inhibitors_2020}.
In cases of discordant results, it helps to look at details and meta-analyses
may include issues like too low a dose, too little lag time or 
confounding factors. Results may vary with population too.
In general, stomach acid is expected to decrease with age
and the use off PPI's iq being questions \cite{Mehta_Guasch_Kamen_Proton_Pump_Inhibitors_2020} they may not have much impact. 
Low stomach acid is just one of many possible conditions
that imprair nutrient uptake in aging populations.
However, it appears to be easy to correct with more
dietary acids such as citric or acetic with meals.
Chloride sources may include potassium chloride with meals.


 


%\begin{comment}
Many suspects
exist as nutritional contributors from amino acids to vitamin K. 
Immediate benefits of simple nutrient additions may
not be apparent until GI and vasculara damage is repaired
or uptake aids are added. 

Reasons are considered why simplistic
interventions have failed and indeed controlled tests
of "vitamins" may not reflect real-world anecdotes.
In any case, an interpretation based on reaction rather
than cause can explain most of the known evidence 
. If there is a specific organism group driving clinical
dementia it still needs to be better defined 
as there may be risks to symbiotics with broad antibiotic treatments. 
Further work can be considered to obtain phenotype information
such as microbial mRNA analysis and microbiomes of
younger brains with known fetal and peri-natal exposure histories.  
"Infection-at-conception" would be an intriguing concept to explore
even if organisms do not persist into old age. 
%to examine mRNA in brain microbiome including younger deceased. 

These results are consistent with 
my earlier work on \mjmdisease suggesting that nutient loss 
due to age related GI  damage is a correctable disease driver 
even if not reflected in blood levels
due to vascular damage and stress transport. 
It is also possible however that prior neuron damage
unrelated to nutrient supply 
caused a decrease in Trp and Tyr mobilization or 5HT
synthesis.
%\end{comment}
%While not considered from a nutritional standpoint before,

\mjmtol{ there is a recurring problem with equating a blood level with
a production rate when the sources and sinks or rate equations are not known. }
Interestingly this may be produced by GI bacteria 
\cite{10.1371/journal.pone.0090239} making them the pathogens effectively
responsible for dementia. 


%%%%%%%%%%%%%%%%%%%%%%%%%%%%%%%%%%%%%%%%%%%%%%%%%%%%%%%%%%%%%

Microbes then could be contributory through many mechanisms.
Besides direct infection of the brain and production of toxic
metabaolites, modulation of nutrient availability could
be achieved by direct metabolism and damage to the GI tract.
The former is documented for choline and \mjmvirus is likely
to cause GI damage possibly related to ACE2 and therefore
tryptophan transport. 
As this likely accumulates with age and reaches a positive
feedback stage where lack of nutirients leads to further
GI decay, it could explain age distributions. 
On the other hand, microbes in the brain make a functional
measure of the brain environment at least for 
bacteria. The ecology may be reflected in OTU abundances
and phenotype information from mRNA of the more flexible
organisms may further help interpretation.
Rate equations accounting for sources and sinks of nutrients
are probably needed in many cases. 






The other insidattors
are not clean measurements of but do point to the
importance of tryptophan and tyrosine avialability
or metabolism. The ability to metabolize iron in the
diseased state is consistent with known vascular issues
and helps validate the results and reaffirm vascular
weakness and maybe a role for vitamin K and better
clotting. 
Prior work by other groups specialized towards
Alzheimer's has been published with encouraging
results although well controlled clinical trials
have not been performed to the best of my knowledge.





My prior work has concentrated on dogs but
required much of the human literature arrving at
 outlines such as  \mjmrefapp{baseline}.
Previously published dog diets are similar
\cite{marchywka-MJM-2022-013-0.10}
\cite{mmarchywka-MJM-2021-003-v0.50rg}
but more specific ones may be out soon.
Recent experience has suggested a role for things such as
benzoate which may improve solubility and uptake
from GI tract of have other benefits\cite{marchywka-MJM-2022-013-0.10}
 and should be
a subject of a future work based on experience with dogs. Nutrient context
will likely matter as much as amount. A concentrated
concern for lipic solubles
may be warranted. 
Additionally, salmon broth and vinegar both appear
to help dissolve components such as hardboiled egg yolk.
For humans, ethanol may also be beneficial explaining
inconsistent health benefits associated with alcohol
consumption as they may only occur when consumed with
appropriate food containing otherwise inaccessible nutrients.
The work with dogs continues to focus on combinations
similar to deep eutectic solvents i\cite{marchywka-MJM-2021-018-0.50rg}.
The association of AD with APOE4
may in fact be related to transport of lipid soluble
nutrients. 

In the case of vascular pathology such as CAA, the best rememdy of
course is to clear the plaques and restore the normal
popylations of transporters. This the CAA may itself be due to
nutrient deficiencies and eventually correct itself
but in the meantime surpluses may achieve similar
results by diffusion. 


Interpretation in part is difficult due to operating through
taxonomy which is a combination of historical obervations 
with some modifications for molecular "closeness."  Phenotype
and even genotype/plasmids  are not known but may be inferred from the
16s sequence and overall ecology likely to host the given
abundance sets. Adding mRNA data  may reduce ambiguity with
expression of nutrient synethesis or aqusition genes or known
lifestyles invoked in known environments. Another approach
may be to look only at the complete 16s sequence reads and
determine if particular fragments seqgrate to disease or
control and if they mean anything about phenotype. 
This is a bit speculative but there is existing literature to
classify organisms based only on 16s patterns.  


\begin{comment}
In previous work, I had motivated the idea that many
afflictions of old age are the result of nutrient deficiencies largely
due to GI problems which are often ignored. Much of this argument
was made in a response to \mjmdisease which continues to
be most severe in the elderly and recently described
as specifically the frail elderly or those most resembling
the picture of classic starvation.
Alzheimer's disease remains as an outstanding age-correlated disease.
While some drug approvals have been predicated on amyloid beta
removal, current trial performance is limited and real-world
experience unclear. A recent work describing the microbiome of
postmortem brain samples distinguished by AD status highlighted
organisms with known relationships to ambient tryptophan, tyrosine or
metabolites such as serotonin or benzene derivatives. 

The ability to do "clean" experiments on biological entities
is largely an illustion. However, even ambiguous data have value
if they are analyzed that way. Correlations may be useful but
any intervention is predicated on some caisality even
if that is never entirely elucidated or even stated.
And of course they fail due to only an imaginary re3ationship.

Microbiomes would unlikely be considered  clean or simple
experiments but given enough organism abundances, and knowledge
of organism lifestyle, some inferences may be made about importat
states or ecology that may not be obvious a priori or compelling
enough to do tests for specific metabolites. Genotypes as
reflected in 16s rRNA and perhaps phenotypes as reflected in 
mRNA sequencing may be good open ended tests but yield in
essence "transform data" similar to say fourier or other techniques.

Some recent work on miRNA pointed to a causal role for MEG3
in neuron loss in AD patients prompting the addition of another
target to attack. However, there is some indication even this
may reflect Trp limitations. 
  

\end{comment}




\section{Conclusion}
The top-line microbiome patterns of the subject paper 
\cite{10.3389/fcimb.2023.1123228} 
can be interpretted as demonstrating a role for symbiotic organisms
acquired as early as conception or the peri-natal period 
combinted with  measurment of  the brain environment properties likely to be
improtant to neuronal functioning. 
The organisms more abundant
in the control brains 
may thrive in setting rich in amino acid derived neutotransmitter
skeletons or 
may be beneficial for plants and consequentially
or coincidnetally have similar relationships to brain 
by aiding nutrient uptake among other functions. They
generally are more competitive in environments presumed
healthy for the brain and maybe with some toxic metals present. 
Those more abundant in the AD brain almost exclusively 
would benefit from persumed deleterious states of 
depeted 5HT, incrased iron ( speculating heme derived ),
and increased methanol from endogenous metabolism. 
Staphylococcus eepidermidis is a notable standout however
as a potential pathogen able to cause brain damage.
Further work should explore larger patterns of possible pathogen
involvement in clinical progression.  
While likely coincidental, to physical similarity between
plant root and axon may be worthy further consideration.
A lot of unrelated results can be unified into a nutritional framework
with tryptophan being one of the preominent recurring components.
Earlier predictions emphasizing Trp, Tyr, and vitamin K
for \mjmdisease  have not been significantly tested but 
evolving evidence exaplored in the light of cause and effect
rather than a spercific coincidence supports their utlity
in old age conditions. Further work on microbial patterns including
metabolic phenotype may be helpful. Nutritional experiments probably
need to be more comprehensive including several nutrients, exlduing
others, and including solubility enhancements perhaps similar
to those employed by the organisms overabundant in the healthy brains.  
And its important to continue to question assumptions just in 
case lead can make your garden grow and feed your head.

\section{Supplemental Information}

\subsection{Computer Code}


\begin{lstlisting}


\end{lstlisting}
\section{Bibliography}


\bibliography{\mjmbasename,\mjmaddbio}
\bibliographystyle{plainurl}


%%%%%%%%%%%%%%%%%%%%%%%%%%%%%%%%%%%%%%%%%%%%%%%%%%%%%%%%%%%%%%%%%%%%%%%%%%%%%
\begin{acknowledgments} 

% \input{generalack.tex}
\begin{enumerate}
\item Nikki Schultek for brining this work \cite{Schultek_Nikki_Schultek_LinkedIn_2023} as well as the infectious hypothesis to my attention. 
\item Pubmed eutils facilities and the basic research it provides. 
\item Free software including Linux, R, LaTex  etc.
\item Thanks everyone who contributed incidental support. 
\item I have to credit my own software such as bash scripts
 and of course TooBibi\cite{mmarchywka-MJM-2021-002-v0.1.1-rg} for facilitating citation discovery. 
\end{enumerate}

\end{acknowledgments}

%%%%%%%%%%%%%%%%%%%%%%%%%%%%%%%%%%%%%%%%%%%%%%%%%%%%%%%%%%%%%%%%%%%%%%%%%%%%%
\clearpage
\appendix


\section{ Statement of Conflicts }
 No specific funding was used in this effort and there are no financial
relationships with others that could create a conflict of interest. 

\section{About the Authors}
This work was performed at a dog rescue run by Barbara Cade and
housed in rural Georgia.  The author of this report 
,Mike Marchywka,
has a background in electrical engineering and 
has done extensive research using free online literature sources.  
I hope to find additional people interested in critically 
examining the results.

\section{Interpretation Issues- General and Specific }
\label{appendix:interp}


However, increased abundance simply indicates comparative fitness
increase. Assuming histidine is limiting, that may be due to
increased supply due to host signalling or a virulent
organism releasing it or failure of senescent neurons to 
compete. All fitness arguments and static levels need to
be carefully considered with rate equations that include
feedback for source control.  

While trying to generate hypotheses about cause and effect
in order to ultiamtely design a useful intervention, its
important to remember some caveats specific to these data
and ambient nutrient levels. 
The features and citations picked out for this work may be
influenced by literature skew towards trendy topics and
selection bias. 
The data are all relative
abundances so no real inferences about absolute
amounts can be made and all the processings
steps employed by the original authors have not been
explored. Much of the analysis will now depend
on reasoning such as "nutrient X favors organism Y"
but the available amount of "X" could itself be
due to many factors. Ideally a rate equation
should be written to account for all sources and sinks
of "X" but even in that case both are likely to be
controlled by feedback mechanisms. Tryptophan along
with biotin have been shown to be exported in
times expected to beneeded by the brain but
transport into the CNS may be impeded. 



The relationship between microbe relative abundance,
nutrient status, and brain health could be quite complicated
even if fitness could be easily related to nutrient
concentration. The limiting nutrient could exist
in amounts determined by various feedback signals
from functioning brain or be limited by supply and
vascular function. High concentration could even reflect
low uptake by neurons in some diseased states. A rate
equation with the right control terms could show the complexity
of infering cause and effect. 




\section{Symbols, Abbreviations and Colloquialisms}

\begin{comment}
% grep "[A-Z][A-Z]" paradox.tex | sed -e 's/[^A-Z]/\n/g' | grep "[A-Z]" | sort | uniq -c
% cat  paradox.tex | sed -e 's/  */\n/g' | grep "[A-Z][A-Z]"  | grep -v "[^A-Z]" | sort | uniq  |awk '{print $0" &   \\\\"; }'
\end{comment}


%\abbreviations{The following abbreviations are used in this manuscript:\\
%\begin{table}
\noindent
\begin{tabular}{@{}ll}
%SMVT & Sodium dependent Multi-Vitamin Transporter\\
TERM & definition and meaning   \\
\hline
%TLA & Three letter acronym\\
%LD & linear dichroism
\end{tabular} % }
%\end{table}

% https://tex.stackexchange.com/questions/5957/bibtex-entry-for-white-papers-and-technical-reports

\section{General caveats and disclaimer }
\label{appendix:caveats}

%\input{disclaimer-informal.tex}

This document was created in the hope it will be interesting to
someone including me by providing information 
about some topic that may include personal experience or a literature
review or description of a speculative theory or idea.
There is no assurance that the content of this work will be
useful for any paricular purpose. 
%In no case am I claiming to provide useful advice on any matter
%but attempting to describe events in terms of literature known
%to me. 


All statements in this document were true to the best of my knowledge
at the time they were made and every attempt is made to assure
they are not misleading or confusing. However, information provided by
others and observations that can be manipulated by unknown causes  
( "gaslighting" ) may be misleading. Any use of this information should
be preceded by validation including replication where feasible.
Errors may enter into the final work at every step from conception
and research to final editing. 
%No assurance can exist that obvious conclusions will be useful
%and may be misleading. 



Documents labelled "NOTES" or "not public" contain
substantial informal or speculative content that
may be terse and poorly edited or even sarcastic or profane.
Documents labelled as "public" have generally been edited
to be more coherent but probably have not been reviewed
or proof read. 

Generally non-public documents are labelled as such to avoid
confusion and embarassment and should be read with that understanding.


\section{ A basline nutrient outline  }
\label{appendix:baseline}
% https://tex.stackexchange.com/questions/231551/using-ifdefined-on-csname-macros
 \ifdefined\mjmstandalone%
%
\else%
\newcommand{\mjmstandalone}[1]{%
  \ifdefined\MJMTEXFRAG%
% \ifcsname\MJMTEXFRAG\endcsname%
   #1%
  \fi%
}
  \fi%


\begin{table}[H] \centering
\begin{tabular}{|l|r|c|l|}
\hline
\multicolumn{4}{c}{Title}\\
\hline
Ingredient & Amount & Relevance & Refs/Notes \\
\hline
%Depakene ( valproate ) & 250mg/day & late onset AD & \\
Lysine HCl  & 2600mg/day  & & rorate K/R  \\
Arginine  HCl  & 1300mg every other  & & rotate K/R \\
\hline
Threonine   & 1300mg/day  & & \\
\hline
Leucine   & 650mg/day  & & rotate BCAA  \\
Valine   & 800mg/day  & & rotate BCAA \\
Isoleucine   & 800mg/ every few days  & & rotate BCAA \\
\hline
Methionine  & 250mg/ day  & & \\
Histidine   & 340mg/ day  & & \\
Tryptophan   & 500mg/ day  & & \\
Phenylalanine   & 500mg/ day  & & not F+Y no BH4 \\
Tyrosine   & 500mg/ day  & & not F+Y no BH4 \\
Glutamine   & 0mg/ day  & & does not seem to help \\
\hline
Taurine  & 1800mg/ day  & & \\
Lecithin   & 1800mg/ day  & & choline and emulsifire  \\
citric acid    &   & &   \\
acetic acid    &   & &   \\
KCl    &   & &   \\
benzoate &&& mix likely matters \\
sorbitol &&&  absorption aid \\
\hline
vitamin K1   &   10mg/day   & & rotate lipid solubles    \\
vitamin K2/MK4/MK7  &   10mg/day   & & mix probably matters    \\
\hline
zinc  &  every other day    & &   \\
magnesium  & 400mg/day    & & see text, citrate+citrate or other   \\
copper  &  5mg every other day    & &   \\
Iodine   &  1mg every few days    & &  rotate SMVT  \\
\hline
\end{tabular}
\caption{}
%\label{}
\end{table}


\begin{table}[H] \centering
\begin{tabular}{|l|r|c|l|}
\hline
\multicolumn{4}{c}{Title}\\
\hline
Ingredient & Amount & Relevance & Refs/Notes \\
\hline
%Depakene ( valproate ) & 250mg/day & late onset AD & \\
%vitamin K2/MK4/MK7  &   10mg/day   & & mix probably matters    \\
B-1   &   100mg/day   & &  \\
B-2   &   400mg/day   & &  reactive, see text  \\
B-3   &   100mg/day   & &   niacin may be better, see text  \\
B-6   &   100mg/day   & &    \\
B-12   &   1mg/ every few days  & & out compete cancer etc     \\
folate   &   1mg/ every few days  & & out compete cancer etc     \\
biotin   &   10mg/day  & & rotate SMVT     \\
calcium pantothenate   &   500mg/day  & & rotate SMVT, beta-alanine ok too      \\
lipoic acid   &   200mg/ every few days  & & rotate SMVT  \\
\hline
vitamin C  &  sporadic  & & avoid benzoate  \\
vitamin A  &  sporadic  & & rotate lipid solubles  \\
vitamin D  &  sporadic  & & rotate lipid solubles  \\
\hline
turkey  & & & \\
chicken thigh  & & & \\
ground beef 7-20 pct fat  & & & \\
shrimp  & & & \\
tuna  & & & \\
salmon  & & & \\
Eggland eggs  & & & \\
spinach  & & & \\
carrot  & & & \\
garlic  & & & \\
EV olive oil   & & & \\
\hline
\end{tabular}
\caption{}
%\label{}
\end{table}




\section{Citing this as a tech report or white paper }
\label{appendix:citing}

Note: This is mostly manually entered and not assured to be error free.

This is tech report \mjmtrno. 

\begin{table}[H] \centering
\begin{tabular}{r|r|c|r}
Version & Date & Comments  &  \\
0.01 & \mjmmakedate  &  Create from empty.tex template  &  \\
-  & \today & version  \mjmversion { }   \mjmtrno  &  \\
1.0 & 20xx-xx-xx & First revision for distribution &  \\
\end{tabular}
\end{table}


Released versions,

build script needs to include empty releases.tex
\begin{table}[H] \centering
\begin{tabular}{|r|r|l|}
Version & Date & URL    \\
\hline
&  &  \\
% version & date & url  \\
%.1 table & 2021-08-17& {\url{https://www.linkedin.com/posts/marchywka_draft-compare-72020-theory-with-interim-activity-6833343119203860480--wJv}} \\
%.1 table & 2021-08-17& {\url{https://www.researchgate.net/publication/353946686_Draft_table_comparing_expectations_to_recent_results_with_covid-19}} \\
%.1 table & 2021-08-17 & {\url{https://www.academia.edu/s/34e160cae9}} \\

\hline
\end{tabular}
\end{table}





% 2020-11-30 keep on same page 
%\input{bibtex2.txt}

\begin{minipage}{\linewidth}
%\input{bibtex2.txt}
%\input{bibtex3.txt}
\mjmshowbib
\end{minipage}





\vspace{1cm}
Supporting files. Note that some dates,sizes, and md5's will change as this is
rebuilt.

This really needs to include the data analysis code 
but right now it is auto generated picking up things from prior
build in many cases 
\lstinputlisting{\mjmbasename.bundle_checksums}
\end{document}
